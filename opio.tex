\documentclass{article}
\pagenumbering{arabic}

\begin{document}
\title {THE EFFECTS OF STRESS ON STUDENTS}
\author{OPIO EMMANUEL OMONA . 15/U/12219/PS . 215016373}
\maketitle
\section{Introduction}
\subsection{Background}
Today, many schools and universities are experiencing the effects of stress on
work performance. The effects of stress can be either positive or negative. What is
perceived as positive stress by one person may be perceived as negative stress by
another, since everyone perceives situations differently. Managers need to identify those
suffering from negative stress and implement programs as a defense against stress.
These programs may reduce the impact stress has on students’ work performance. 

\subsection{Statement of the Problem}
The purpose of this study was to determine the negative effects of stress on students
and the methods the administration can use to manage students’ stress.

\section{Significance of the Study}
There are three primary groups that may benefit from this study. The first group,
consisting of students in today's universities, may learn to identify ways
that stress negatively affects their work performance. Identifying the negative effects
may enable them to take necessary action to cope with stress. By sharing this
knowledge, students can act as a vehicle to help management implement appropriate
stress reduction programs.  The second group that may benefit from this study is
administration who may gain insight as to how stress is actually negatively affecting students’ work performance. Finally, educators can use these findings as a valuable
guide to incorporate into their curriculum. By emphasizing to students the importance
of developing programs to deal with stress, the students may be able to transfer this
knowledge to the workplace, thereby improving the quality of the work environment.

\section{Scope of the Study}
This study was limited to the perceptions of full-time business employees as to the
negative effects that stress has on work performance and the steps that employers are
taking to manage stress. For the purpose of this study, what constitutes full-time
employment is defined by the employer. This study was restricted to businesses
operating in the Central Texas area. The Central Texas area encompasses all
communities within Hays, Kendall, Travis, and Williamson counties. For the purpose
of this study, stress is defined as disruptive or disquieting influences that negatively
affect an individual in the workplace/school. 

\section {Methods of the Study}
The data on this topic was got from a number of sources that include the writer who is a student at Makerere University, his fellow students and even some of the past students from other Universities.

\subsection{Causes of Stress}
There are many causes of stress however this research will only focus on two main points
\begin{enumerate}
\item The firsts is that students are sometimes given a lot of work to do in a short period of time.
\item The second is that students do not always agree on some of the University policies like Tuition and Exam policies to mention but a few.
\end{enumerate}

\subsection{Effects On Students}
Some of the effects stress has on students can be found below:
\begin{enumerate}
\item Students can cause failure in academics due to reading on tension.
\item Students have also been found to lose weight because of over thinking when there is a lot due work.
\item Stress also causes students to look for an easy way out of problems and this makes them strike over issues that can be solved otherwise.
\end{enumerate}

\section{Conclusion}
I conclude by saying all factors that causes stress have not been indicated in this study. Although the few that have been mentioned have been by concrete evidence of examples.
Also all the effects have not been included due to the small sample of students that was got.

\end{document}
